\begin{frame}

\tikzstyle{every picture}+=[remember picture]

%\frametitle{Functions}
\note{
A mathematical \textbf{function} is a relation that maps from input values $x$ to output values $y$.
%
To keep track of various functions, by convention each function is given a name.
%
\\ \ \\

In this example, we have a function called $f$.
%
The function $f$ accepts a single input parameter.
%
For convenience, input parameters are also named; here our input parameter is called $x$.
%
In this example, the value of the input parameter $x$ is constrained to be any one letter from the set of letters in the English alphabet.
%
Given a concrete value for the input parameter $x$, applying the function $f$ will result in a value from the set \{vowel,consonant\} being assigned to $y$.
%
\\ \ \\

Let's walk through an example.
%
If we apply the function $f$ to the input \textit{a}, the result is \textit{vowel}.
%
If we instead supply \textit{b} as the input, the result of the function is \textit{consonant}.
%
For every possible value of $x$, exactly one possible value of $y$ is returned.
%
In this example, the function returns \textit{vowel} for the input values \textit{a}, \textit{e}, \textit{i}, \textit{o}, and \textit{u}, and returns \textit{consonant} for the remaining values of $x$.

}

%\begin{exampleblock}
\begin{center}
$\tikzmark{functionOutput}y=\tikzmark{functionName}f(\tikzmark{functionInput}x)$
\end{center}
%\end{exampleblock}


\begin{columns}[T] % align columns
\begin{column}{.48\textwidth}
\begin{center}

Possible input values of $x$

{\color{blue}\rule{\linewidth}{4pt}}

\ \\ 

\begin{tabular}{|c c c c|}
\hline
\tikz[baseline]{\node[anchor=base] (a) {a};} & \tikz[baseline]{\node[anchor=base] (h) {h};} & \tikz[baseline]{\node[anchor=base] (o) {o};} & \tikz[baseline]{\node[anchor=base] (v) {v};} \\
\tikz[baseline]{\node[anchor=base] (b) {b};} & \tikz[baseline]{\node[anchor=base] (i) {i};} & \tikz[baseline]{\node[anchor=base] (p) {p};} & \tikz[baseline]{\node[anchor=base] (w) {w};} \\
\tikz[baseline]{\node[anchor=base] (c) {c};} & \tikz[baseline]{\node[anchor=base] (j) {j};} & \tikz[baseline]{\node[anchor=base] (q) {q};} & \tikz[baseline]{\node[anchor=base] (x) {x};} \\
\tikz[baseline]{\node[anchor=base] (d) {d};} & \tikz[baseline]{\node[anchor=base] (k) {k};} & \tikz[baseline]{\node[anchor=base] (r) {r};} & \tikz[baseline]{\node[anchor=base] (y) {y};} \\
\tikz[baseline]{\node[anchor=base] (e) {e};} & \tikz[baseline]{\node[anchor=base] (l) {l};} & \tikz[baseline]{\node[anchor=base] (s) {s};} & \tikz[baseline]{\node[anchor=base] (z) {z};} \\
\tikz[baseline]{\node[anchor=base] (f) {f};} & \tikz[baseline]{\node[anchor=base] (m) {m};} & \tikz[baseline]{\node[anchor=base] (t) {t};} &                                              \\
\tikz[baseline]{\node[anchor=base] (g) {g};} & \tikz[baseline]{\node[anchor=base] (n) {n};} & \tikz[baseline]{\node[anchor=base] (u) {u};} &                                              \\
\hline
\end{tabular}
\end{center}
\end{column}%
\hfill%
\begin{column}{.48\textwidth}
\begin{center}

Possible output values of $y$

{\color{red}\rule{\linewidth}{4pt}}

\ \\

\begin{tabular}{|c|}

\hline
\tikz[baseline]{\node[anchor=base] (vowel) {vowel};} \\
\tikz[baseline]{\node[anchor=base] (consonant) {consonant};} \\
\hline
\end{tabular}
\end{center}
\end{column}%
\end{columns}


\begin{tikzpicture}[
  remember picture,
  overlay,
  expl/.style={draw=orange,fill=orange!30,rounded corners,text width=3cm},
  arrow/.style={red!80!black,ultra thick,->,>=latex}
]
\node[below=10mm of pic cs:functionName] (pointer) {};
%\node<.>[draw=green!80!black,ultra thick, fit=(ptd),inner sep=-0.8mm,circle] {};
\onslide<+>{}
\draw<+>[arrow] (pointer) to ([xshift=0.5ex,yshift=-0.75ex]{pic cs:functionName});
\draw<+>[arrow] (pointer) to ([xshift=0.5ex,yshift=-0.75ex]{pic cs:functionInput});
\draw<+>[arrow] (pointer) to ([xshift=0.5ex,yshift=-0.75ex]{pic cs:functionOutput});

\draw<+>[arrow] (pointer) to ([xshift=0.5ex,yshift=-0.75ex]{pic cs:functionName}); 
\draw<+>[arrow] (pointer) to ([xshift=0.5ex,yshift=-0.75ex]{pic cs:functionInput});
\draw<+>[arrow] (pointer) to ([xshift=0.5ex,yshift=-0.75ex]{pic cs:functionOutput});
\onslide<+>{}
\draw<+>[->,green!80!black,ultra thick] (a) to[out=0,in=180] (vowel);       \node<.>[draw=green!80!black,ultra thick, fit=(a),inner sep=-0.8mm,circle] {};
\draw<+>[->,green!80!black,ultra thick] (b) to[out=0,in=180] (consonant);   \node<.>[draw=green!80!black,ultra thick, fit=(b),inner sep=-0.8mm,circle] {};


\draw<+>[->,green!80!black,ultra thick] (a) to[out=0,in=180] (vowel);       \node<.>[draw=green!80!black,ultra thick, fit=(a),inner sep=-0.8mm,circle] {};
\draw<+>[->,green!80!black,ultra thick] (e) to[out=0,in=180] (vowel);       \node<.>[draw=green!80!black,ultra thick, fit=(e),inner sep=-0.8mm,circle] {};
\draw<+>[->,green!80!black,ultra thick] (i) to[out=0,in=180] (vowel);       \node<.>[draw=green!80!black,ultra thick, fit=(i),inner sep=-0.8mm,circle] {};
\draw<+>[->,green!80!black,ultra thick] (o) to[out=0,in=180] (vowel);       \node<.>[draw=green!80!black,ultra thick, fit=(o),inner sep=-0.8mm,circle] {};
\draw<+>[->,green!80!black,ultra thick] (u) to[out=0,in=180] (vowel);       \node<.>[draw=green!80!black,ultra thick, fit=(u),inner sep=-0.8mm,circle] {};

\draw<+>[->,green!80!black,ultra thick] (c) to[out=0,in=180] (consonant);   \node<.>[draw=green!80!black,ultra thick, fit=(c),inner sep=-0.8mm,circle] {};
\draw<+>[->,green!80!black,ultra thick] (d) to[out=0,in=180] (consonant);   \node<.>[draw=green!80!black,ultra thick, fit=(d),inner sep=-0.8mm,circle] {};
\draw<+>[->,green!80!black,ultra thick] (f) to[out=0,in=180] (consonant);   \node<.>[draw=green!80!black,ultra thick, fit=(f),inner sep=-0.8mm,circle] {};
\draw<+>[->,green!80!black,ultra thick] (g) to[out=0,in=180] (consonant);   \node<.>[draw=green!80!black,ultra thick, fit=(g),inner sep=-0.8mm,circle] {};
\draw<+>[->,green!80!black,ultra thick] (h) to[out=0,in=180] (consonant);   \node<.>[draw=green!80!black,ultra thick, fit=(h),inner sep=-0.8mm,circle] {};
\draw<+>[->,green!80!black,ultra thick] (j) to[out=0,in=180] (consonant);   \node<.>[draw=green!80!black,ultra thick, fit=(j),inner sep=-0.8mm,circle] {};
\draw<+>[->,green!80!black,ultra thick] (k) to[out=0,in=180] (consonant);   \node<.>[draw=green!80!black,ultra thick, fit=(k),inner sep=-0.8mm,circle] {};
\draw<+>[->,green!80!black,ultra thick] (l) to[out=0,in=180] (consonant);   \node<.>[draw=green!80!black,ultra thick, fit=(l),inner sep=-0.8mm,circle] {};
\draw<+>[->,green!80!black,ultra thick] (m) to[out=0,in=180] (consonant);   \node<.>[draw=green!80!black,ultra thick, fit=(m),inner sep=-0.8mm,circle] {};
\draw<+>[->,green!80!black,ultra thick] (n) to[out=0,in=180] (consonant);   \node<.>[draw=green!80!black,ultra thick, fit=(n),inner sep=-0.8mm,circle] {};
\draw<+>[->,green!80!black,ultra thick] (p) to[out=0,in=180] (consonant);   \node<.>[draw=green!80!black,ultra thick, fit=(p),inner sep=-0.8mm,circle] {};
\draw<+>[->,green!80!black,ultra thick] (q) to[out=0,in=180] (consonant);   \node<.>[draw=green!80!black,ultra thick, fit=(q),inner sep=-0.8mm,circle] {};
\draw<+>[->,green!80!black,ultra thick] (r) to[out=0,in=180] (consonant);   \node<.>[draw=green!80!black,ultra thick, fit=(r),inner sep=-0.8mm,circle] {};
\draw<+>[->,green!80!black,ultra thick] (s) to[out=0,in=180] (consonant);   \node<.>[draw=green!80!black,ultra thick, fit=(s),inner sep=-0.8mm,circle] {};
\draw<+>[->,green!80!black,ultra thick] (t) to[out=0,in=180] (consonant);   \node<.>[draw=green!80!black,ultra thick, fit=(t),inner sep=-0.8mm,circle] {};
\draw<+>[->,green!80!black,ultra thick] (v) to[out=0,in=180] (consonant);   \node<.>[draw=green!80!black,ultra thick, fit=(v),inner sep=-0.8mm,circle] {};
\draw<+>[->,green!80!black,ultra thick] (w) to[out=0,in=180] (consonant);   \node<.>[draw=green!80!black,ultra thick, fit=(w),inner sep=-0.8mm,circle] {};
\draw<+>[->,green!80!black,ultra thick] (x) to[out=0,in=180] (consonant);   \node<.>[draw=green!80!black,ultra thick, fit=(x),inner sep=-0.8mm,circle] {};
\draw<+>[->,green!80!black,ultra thick] (y) to[out=0,in=180] (consonant);   \node<.>[draw=green!80!black,ultra thick, fit=(y),inner sep=-0.8mm,circle] {};
\draw<+>[->,green!80!black,ultra thick] (z) to[out=0,in=180] (consonant);   \node<.>[draw=green!80!black,ultra thick, fit=(z),inner sep=-0.8mm,circle] {};
\end{tikzpicture}


\end{frame}


\begin{frame}[fragile]

\begin{minted}{swift}
    enum EnglishLetter {
      case a, b, c, d, e, f, g, h, i, j, k, l, m, 
           n, o, p, q, r, s, t, u, v, w, x, y, z
    }
    
    enum SpeechSound { case vowel, consonant }
    
    func f(x: EnglishLetter) -> SpeechSound {
      
      if (x==EnglishLetter.a || x==EnglishLetter.e || 
          x==EnglishLetter.i || x==EnglishLetter.o || 
          x==EnglishLetter.u) {     
        return SpeechSound.vowel        
      } else {        
        return SpeechSound.consonant        
      }
      
    }

\end{minted}


\note{Now let's consider how this function might be implemented in a traditional programming language.
%
For this example, we'll use the Swift programming language, but the ideas shown here should equally apply to most other programming languages.
%
\\ \ \\

Let's begin by defining a type that enumerates the letters in the English alphabet:
%
enum EnglishLetter \{
%
case a, b, c, d, e, f, g, h, i, j, k, l, m, n, o, p, q, r, s, t, u, v, w, x, y, z \}
%
Next, let's define a type that enumerates the legal output values:
%
enum SpeechSound \{
%
case vowel, consonant \}
%
Given these two types, we now define a function \textit{f} that accepts an input parameter \textit{x} of type \textit{EnglishLetter} and which returns a value of type \textit{SpeechSound}.
%
If $x$ has the value \textit{a} or the value \textit{e} or the value \textit{i} or the value \textit{i} or the value \textit{o} or the value \textit{u}, the function \textit{f} should return \textit{vowel}.
%
Otherwise, the function should return \textit{consonant}.
%
\\ \ \\

Once the function has been defined, we can call the function with various input values.
%
When called with $x$ set to \textit{a}, the function returns the value \textit{vowel}.
When called with $x$ set to \textit{e}, the function returns the value \textit{vowel}.
When called with $x$ set to \textit{q}, the function returns the value \textit{consonant}.
When called with $x$ set to \textit{b}, the function returns the value \textit{consonant}.

}

\end{frame}

