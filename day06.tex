% !TEX TS-program = xelatex --shell-escape
% !TEX encoding = UTF-8 Unicode
%
\documentclass{beamer}
%\setbeamertemplate{items}[ball] 
\DeclareMathOperator*{\argmax}{arg\,max}
\usecolortheme{whale}
\useinnertheme[shadow]{rounded}
\usenavigationsymbolstemplate{}
\usepackage[subsection=false,headline=empty,footline=outlineauthortitle]{beamerouterthememiniframesbottom}
%\usepackage{beamerthemesplit}
\usepackage{tikz}
\usetikzlibrary{arrows,automata,positioning,shapes,shapes.geometric,fit}
\usefonttheme[onlymath]{serif}

\usepackage{enumitem}

\usepackage{textcomp}
% Set up citation style
\usepackage{natbib}
\usepackage{bibentry}
\bibpunct{(}{)}{;}{a}{,}{,}
\newcommand{\newcite}[1]{\citet{#1}}
\renewcommand{\cite}[1]{\citep{#1}}

%\setbeamercovered{transparent}

% Specify that no bibliography should be printed
\bibliographystyle{plainnat}

\usepackage{pgfplots}

\usepackage{mathtools}

\newcommand\defeq{\stackrel{\mathclap{\normalfont\mbox{\tiny def}}}{=}}
\newcommand{\Lim}[1]{\raisebox{0.5ex}{\scalebox{0.8}{$\displaystyle \lim_{#1}\;$}}}

%\usepackage{ dsfont }
%\setbeamertemplate{note page}[plain]
%\setbeameroption{show notes on second screen=right}
%\usepackage{minted}
\title{Neural Network Concepts}
%\input{author}
%\institute[shortinst]{University of Illinois at Urbana-Champaign}

\author{Differential calculus}

\date{}

\begin{document}

\frame{
\titlepage

}

\frame{

%\begin{axis}[
%    axis lines=middle
%]
%\addplot {x^2+rand};
%\end{axis}
\begin{center}
\begin{tikzpicture}
	\begin{axis}[width=190pt,axis x line=middle, axis y line=center]
%		\addplot+[mark=none,smooth] (\x,{sin(\x r)});
%		\addplot {x};
 		\addplot[blue] {2*x + 1};
	\end{axis}
\end{tikzpicture}

\ \\

\ \\

\onslide<2->{y = }$f(x) = mx + b$
\pause

\ \\

\only<3>{$m = \frac{y - b}{x} $}

\only<4>{$\forall_{(x_1, y_1), (x_2, y_2)} \ \ \ m = \frac{y_2 - y_1}{x_2 - x_1}$}

\only<5>{$m = \frac{\Delta y}{\Delta x}$}

\only<6>{$m = \frac{f(x + \Delta x) - f(x) }{\Delta x}$}

\end{center}

}


\frame{

\begin{center}

\only<1>{
\begin{tikzpicture}
\pgfplotsset{xmin=-2, xmax=2, ymin=-1.5, ymax=1.5}
	\begin{axis}[width=190pt,axis x line=middle, axis y line=center]
 		\addplot[blue,smooth] {sin(deg(x))};
	\end{axis}
\end{tikzpicture}
}

\only<2>{
\begin{tikzpicture}
\pgfplotsset{xmin=-2, xmax=2, ymin=-1.5, ymax=1.5}
	\begin{axis}[width=190pt,axis x line=middle, axis y line=center]
 		\addplot[blue,smooth] {sin(deg(x))};
		\addplot[red,mark=*] coordinates {(1.5707963267948966,0)};
	\end{axis}
\end{tikzpicture}
}

\only<3>{
\begin{tikzpicture}
\pgfplotsset{xmin=-2, xmax=2, ymin=-1.5, ymax=1.5}
	\begin{axis}[width=190pt,axis x line=middle, axis y line=center]
 		\addplot[blue,smooth] {sin(deg(x))};
		\addplot[red,mark=*] coordinates {(1.5707963267948966,0)};
		\addplot[red,smooth] {1)};
	\end{axis}
\end{tikzpicture}
}

\only<4>{
\begin{tikzpicture}
\pgfplotsset{xmin=-2, xmax=2, ymin=-1.5, ymax=1.5}
	\begin{axis}[width=190pt,axis x line=middle, axis y line=center]
 		\addplot[blue,smooth] {sin(deg(x))};
		\addplot[red,mark=*] coordinates {(0,0)};
	\end{axis}
\end{tikzpicture}
}

\only<5>{
\begin{tikzpicture}
\pgfplotsset{xmin=-2, xmax=2, ymin=-1.5, ymax=1.5}
	\begin{axis}[width=190pt,axis x line=middle, axis y line=center]
 		\addplot[blue,smooth] {sin(deg(x))};
		\addplot[red,mark=*] coordinates {(0,0)};
		\addplot[red,smooth] {x)};
	\end{axis}
\end{tikzpicture}
}

\only<6-8>{
\begin{tikzpicture}
\pgfplotsset{xmin=-2, xmax=2, ymin=-1.5, ymax=1.5}
	\begin{axis}[width=190pt,axis x line=middle, axis y line=center]
 		\addplot[blue,smooth] {sin(deg(x))};
		\addplot[red,mark=*] coordinates {(1.5707963267948966,0)};
		\addplot[red,smooth] {x)};
		\addplot[red,mark=*] coordinates {(0,0)};
		\addplot[red,smooth] {1)};
		\addplot[red!50!black,smooth] {cos(deg(x)))};
	\end{axis}
\end{tikzpicture}
}



\ \\

\ \\

$f \only<7>{= sin}$

\ \\

$D(f) = f^{\prime}\only<7>{ = cos}\only<8>{ = \Lim{\Delta x \rightarrow 0} \frac{f(x + \Delta x) - f(x) }{\Delta x}}$

\end{center}


}

\frame{

Function $f$ takes a single real-valued argument $x$

\begin{equation}
\textrm{\textbf{func}}\ \ f(x:\mathrm{I\!R}) \rightarrow \mathrm{I\!R}
\end{equation}

\ \\

\ \\

Function $D$ takes a function $f$ and produces a new function $f^{\prime}$

\begin{equation}
\textrm{\textbf{func}}\ \ D(f: \left\{ \mathrm{I\!R} \rightarrow \mathrm{I\!R} \right\}) \rightarrow \left\{ \mathrm{I\!R} \rightarrow \mathrm{I\!R} \right\}
\end{equation}

}


\frame{

Function $D$ takes a function $f$ and produces a new function $f^{\prime}$

\begin{equation}
\textrm{\textbf{func}}\ \ D(f: \left\{ \mathrm{I\!R} \rightarrow \mathrm{I\!R} \right\}) \rightarrow \left\{ \mathrm{I\!R} \rightarrow \mathrm{I\!R} \right\}
\end{equation}

\ \\

\pause

\begin{equation}
f(x) = y
\end{equation}

\pause

\begin{equation}
\frac{dy}{dx} \  \defeq \  D(f) = f^{\prime}
\end{equation}

\pause

\begin{equation}
\frac{d}{dx} \  \defeq \  \frac{dy}{dx}
\end{equation}

}


\frame{
\begin{center}

Trigonometric functions

\ \\

\begin{tabular}{|c|c|}
\hline
Function $f$ & Derivative $f^{\prime}$ \\
\hline
\hline
$\sin(x)$ & $\cos(x)$ \\ \hline
$\cos(x)$ & $-\sin(x)$ \\ \hline
$\tan(x)$ & $\sec(x) \times \sec(x)$ \\ \hline
$\csc(x)$ & $-\csc(x) \times \cot(x)$ \\ \hline
$\sec(x)$ & $\sec(x) \times \tan(x)$ \\ \hline
$\cot(x)$ & $-\csc(x) \times \csc(x)$ \\ \hline
\end{tabular}
\end{center}

}


\frame{
\begin{center}

Inverse trigonometric functions

\ \\

\begin{tabular}{|c|c|}
\hline
Function $f$ & Derivative $f^{\prime}$ \\
\hline
\hline
$\arcsin(x)$ & $\frac{1}{\sqrt{1 - x^2}}$ \\ \hline
$\arccos(x)$ & $\frac{-1}{\sqrt{1 - x^2}}$ \\ \hline
$\arctan(x)$ & $\frac{1}{1 + x^2}$ \\ \hline
$\textrm{arccsc}(x)$ & $\frac{-1}{|x|\sqrt{x^2 - 1}}$ \\ \hline
$\textrm{arcsec}(x)$ & $\frac{1}{|x|\sqrt{x^2 - 1}}$ \\ \hline
$\textrm{arccot}(x)$ & $\frac{-1}{1 + x^2}$ \\ \hline
\end{tabular}
\end{center}

}




\frame{
\begin{center}

Hyperbolic functions

\ \\

\begin{tabular}{|c|c|}
\hline
Function $f$ & Derivative $f^{\prime}$ \\
\hline
\hline
$\sinh(x)$ & $\cosh(x)$ \\ \hline
$\cosh(x)$ & $\sinh(x)$ \\ \hline
$\tanh(x)$ & $1 - \tanh(x) \times \tanh(x)$ \\ \hline
$\textrm{csch}(x)$ & $-\coth(x) \times \textrm{csch}(x)$ \\ \hline
$\textrm{sech}(x)$ & $-\tanh(x) \times \textrm{sech}(x)$ \\ \hline
$\textrm{coth}(x)$ & $1 - \textrm{coth}(x) \times \textrm{coth}(x)$ \\ \hline
\end{tabular}
\end{center}

}

\frame{

\begin{center}

\begin{tikzpicture}
\pgfplotsset{xmin=-2, xmax=2, ymin=-1.5, ymax=1.5}
	\begin{axis}[width=190pt,axis x line=middle, axis y line=center]
 		\addplot[blue,smooth] {0.3};
	\end{axis}
\end{tikzpicture}

\ \\

\begin{tabular}{|c|c|}
\hline
Function $f$ & Derivative $f^{\prime}$ \\
\hline
\hline
$f(x) = c$ & $f^{\prime}(x) = 0$ \\ \hline
\end{tabular}

\end{center}

}


\frame{

\begin{center}

\begin{tikzpicture}
\pgfplotsset{xmin=-2, xmax=2, ymin=-1.5, ymax=1.5}
	\begin{axis}[width=190pt,axis x line=middle, axis y line=center]
 		\addplot[blue,smooth] {x};
	\end{axis}
\end{tikzpicture}

\ \\

\begin{tabular}{|c|c|}
\hline
Function $f$ & Derivative $f^{\prime}$ \\
\hline
\hline
$f(x) = x$ & $f^{\prime}(x) = 1$ \\ \hline
\end{tabular}

\end{center}

}



\frame{

\begin{center}

\begin{tikzpicture}
\pgfplotsset{xmin=-2, xmax=2, ymin=-1.5, ymax=1.5}
	\begin{axis}[width=190pt,axis x line=middle, axis y line=center]
 		\addplot[blue,smooth] {x^2};
	\end{axis}
\end{tikzpicture}

\ \\

\begin{tabular}{|c|c|}
\hline
Function $f$ & Derivative $f^{\prime}$ \\
\hline
\hline
$f(x) = x^2$ & $f^{\prime}(x) = 2x$ \\ \hline
\end{tabular}

\end{center}

}


\frame{

\begin{center}

\begin{tikzpicture}
\pgfplotsset{xmin=-2, xmax=2, ymin=-1.5, ymax=1.5}
	\begin{axis}[width=190pt,axis x line=middle, axis y line=center]
 		\addplot[blue,smooth] {x^3};
	\end{axis}
\end{tikzpicture}

\ \\

\begin{tabular}{|c|c|}
\hline
Function $f$ & Derivative $f^{\prime}$ \\
\hline
\hline
$f(x) = x^n$ & $f^{\prime}(x) = n \times x^{n-1}$ \\ \hline
\end{tabular}

\end{center}

}


\frame{

\begin{center}

\begin{tikzpicture}
%\pgfplotsset{xmin=-5, xmax=100, ymin=-2, ymax=1000}
	\begin{axis}[width=190pt,axis x line=middle, axis y line=center]
 		\addplot[blue] {pow(2,x)};
	\end{axis}
\end{tikzpicture}

\ \\

\begin{tabular}{|c|c|}
\hline
Function $f$ & Derivative $f^{\prime}$ \\
\hline
\hline
$f(x) = c^x$ & $f^{\prime}(x) = c^x \times \ln(c)$ \\ \hline
\end{tabular}

\end{center}

}


\frame{

\begin{center}

\begin{tikzpicture}
%\pgfplotsset{xmin=-5, xmax=100, ymin=-2, ymax=1000}
	\begin{axis}[width=190pt,axis x line=middle, axis y line=center]
 		\addplot[blue] {exp(x)};
	\end{axis}
\end{tikzpicture}

\ \\

\begin{tabular}{|c|c|}
\hline
Function $f$ & Derivative $f^{\prime}$ \\
\hline
\hline
$f(x) = e^x$ & $f^{\prime}(x) = e^x$ \\ \hline
\end{tabular}

\end{center}

}

\frame{

\begin{center}

\begin{tikzpicture}
%\pgfplotsset{xmin=-5, xmax=5, ymin=-5, ymax=5}
	\begin{axis}[width=190pt,axis x line=middle, axis y line=center]
 		\addplot[blue] {ln(x)};
	\end{axis}
\end{tikzpicture}

\ \\

\begin{tabular}{|c|c|}
\hline
Function $f$ & Derivative $f^{\prime}$ \\
\hline
\hline
$f(x) = ln(x)$ & $f^{\prime}(x) = \frac{1}{x}$ \\ \hline
\end{tabular}

\end{center}

}

\frame{

\begin{center}

\begin{tikzpicture}
\pgfplotsset{xmin=-5, xmax=5, ymin=-5, ymax=5}
	\begin{axis}[width=190pt,axis x line=middle, axis y line=center]
 		\addplot[blue] {x*2};
	\end{axis}
\end{tikzpicture}

\ \\

\begin{tabular}{|c|c|}
\hline
Function $f$ & Derivative $f^{\prime}$ \\
\hline
\hline
$f(x) = c \times x$ & $f^{\prime}(x) = c$ \\ \hline
\end{tabular}

\end{center}

}


\frame{

What about functions with higher arity?

\ \\

\begin{equation}
f(x_1,x_2) = x_1 + x_2
\end{equation}

}



\begin{frame}

\note{
This presentation was created and narrated by Lane Schwartz. \\ \ \\
You are free to reproduce and adapt this work under the terms of the Creative Commons Attribution-ShareAlike 4.0 International License.
}

\begin{center}
\includegraphics[scale=0.3]{images/CreativeCommons} \ \ \ \ \ 
\includegraphics[scale=0.3]{images/CreativeCommonsAttribution} \ \ \ \ \ 
\includegraphics[scale=0.3]{images/CreativeCommonsSharealike}
\end{center}

\ \\

%\setbeamertemplate{itemize items}[$+$]
\begin{itemize}
\item[\textbullet] This presentation was created and narrated by Lane Schwartz. \\ \ \\
\item[\textbullet] You are free to reproduce and adapt this work under the terms of the Creative Commons Attribution-ShareAlike 4.0 International License.
\end{itemize}

\end{frame}


%\begin{frame}
\frametitle{Linear unit}

\begin{equation*}
y = b + \sum_i x_i w_i
\end{equation*}


\begin{tikzpicture}
\begin{axis}[ylabel=output,xlabel=weighted input,scale=0.3,grid=major,ytick={0},xtick={0},xmin=-5,xmax=5,ymin=-5,ymax=5]
  \addplot[blue, ultra thick][domain=-10:10] (x,x);
\end{axis}
\end{tikzpicture}

\end{frame}


%\input{tmp}


\end{document}
