% !TEX TS-program = xelatex --shell-escape
% !TEX encoding = UTF-8 Unicode
%
\documentclass{beamer}
%\setbeamertemplate{items}[ball] 
\DeclareMathOperator*{\argmax}{arg\,max}
\usecolortheme{whale}
\useinnertheme[shadow]{rounded}
\usenavigationsymbolstemplate{}
\usepackage[subsection=false,headline=empty,footline=outlineauthortitle]{beamerouterthememiniframesbottom}
%\usepackage{beamerthemesplit}
\usepackage{tikz}
\usetikzlibrary{arrows,automata,positioning,shapes,shapes.geometric,fit}
\usefonttheme[onlymath]{serif}

\usepackage{enumitem}

\usepackage{textcomp}
% Set up citation style
\usepackage{natbib}
\usepackage{bibentry}
\bibpunct{(}{)}{;}{a}{,}{,}
\newcommand{\newcite}[1]{\citet{#1}}
\renewcommand{\cite}[1]{\citep{#1}}

%\setbeamercovered{transparent}

% Specify that no bibliography should be printed
\bibliographystyle{plainnat}

\usepackage{pgfplots}

%\usepackage{ dsfont }
%\setbeamertemplate{note page}[plain]
%\setbeameroption{show notes on second screen=right}
%\usepackage{minted}
\title{Neural Network Concepts}
%\input{author}
%\institute[shortinst]{University of Illinois at Urbana-Champaign}

\author{Differential calculus}

\date{}

\begin{document}

\frame{
\titlepage

}

\frame{

\begin{center}

Chain rule

\ \\

\begin{align*}
x &= \ldots \\
y &= f(x) \\
z &= g(y)
\end{align*}

\end{center}

}


\frame{

\begin{center}

Let $f$ and $g$ each be functions with signature $\mathrm{I\!R} \rightarrow \mathrm{I\!R}$

\pause
\ \\

Now, define:

\begin{equation}
h(x) = f(x) + g(x)
\end{equation}

\end{center}

}


\frame{

Definition of derivative:

\begin{align*}
f^{\prime}(x) \defeq \Lim{\Delta x \rightarrow 0} \frac{f(x + \Delta x) - f(x) }{\Delta x}
\end{align*}

}

\frame{ 

\begin{align*}
h(x) = f(x) + g(x)
\end{align*}

\begin{align*}
\onslide<2->{h^{\prime}(x) &= \Lim{\Delta x \rightarrow 0} \frac{h(x + \Delta x) - h(x) }{\Delta x} \\ }
\onslide<3->{ &= \Lim{\Delta x \rightarrow 0} \frac{\left[ f(x + \Delta x) + g(x + \Delta x) \right]  - [f(x) + g(x)] }{\Delta x} \\ }
\onslide<4->{ &= \Lim{\Delta x \rightarrow 0} \frac{\left[ f(x + \Delta x) + g(x + \Delta x) \right]  - f(x) - g(x)] }{\Delta x}  \\ }
\onslide<5->{ &= \Lim{\Delta x \rightarrow 0} \frac{[ f(x + \Delta x) - f(x)] + [ g(x + \Delta x) - g(x) ] }{\Delta x}  \\ }
\onslide<6->{ &= \Lim{\Delta x \rightarrow 0} \frac{f(x + \Delta x) - f(x)}{\Delta x} + \Lim{\Delta x \rightarrow 0} \frac{g(x + \Delta x) - g(x)}{\Delta x}  \\ }
\onslide<7->{h^{\prime}(x) &= f^{\prime}(x) + g^{\prime}(x) \\ }
\end{align*}

}

\frame{ 

\begin{align*}
h(x) &= f(x) + g(x) \\
h^{\prime}(x) &= f^{\prime}(x) + g^{\prime}(x)
\end{align*}

\pause
Now, let $c$ be a constant, and define $f$ and $g$ as:
\pause
\begin{align*}
f(x) &= x \ & g(x) &= c\\
f^{\prime}(x) &= 1   & g^{\prime}(x) &= 0 \ 
\end{align*}

\pause
Therefore

\begin{align*}
h(x) &= f(x) + g(x) \\
     &= x + c \\
h^{\prime}(x) &= f^{\prime}(x) + g^{\prime}(x) \\
              &= 1 + 0 \\
 &= 1 \\
\end{align*}

}

\frame{

\begin{align*}
f(x) &= x + c \\
f^{\prime}(x) &= 1 \\
\end{align*}

}

\frame{

Definition of partial derivative:

\begin{align*}
f_{x_i}^{\prime}(x_1, \ldots, x_{i-1}, x_i, x_{i+1} \ldots, x_n) \\ \ \\
\frac{\partial}{\partial x_i} f(x_1, \ldots, x_{i-1}, x_i, x_{i+1} \ldots, x_n)
%\defeq \Lim{\Delta x \rightarrow 0} \frac{f(x + \Delta x) - f(x) }{\Delta x}
\end{align*}

\begin{align*}
f_{x_i} &= \frac{\partial f}{\partial x_i} \defeq \\ \ \\
& \Lim{\Delta x_i \rightarrow 0} \frac{f(x_1, \ldots, x_{i-1}, x_i  + \Delta x_i, x_{i+1} \ldots, x_n) - f(x_1, \ldots, x_n) }{\Delta x_i}
\end{align*}

}


\frame{

\begin{align*}
f(x_1, x_2) &= x_1 + x_2 \\
f_{x_1}^{\prime}(x_1, x_2) &= 1 \\
\frac{\partial}{\partial x} f(x_1, x_2) &= 1 \\
\end{align*}

}

\frame{


  \begin{center}
  \begin{tikzpicture}
    [neuron/.style={circle,draw=black,fill=white,inner sep=0pt,minimum size=10mm,outer sep=0.5pt},
    >=stealth',
    scale=0.7,transform shape]
    
    \node          (i01)                              {$x_1$};
    \node          (b00)      [above=10mm of i01]     {$1$};
    
    \node [neuron] (n01)      [right=20mm of i01]     {$\tanh$};
%    \node [neuron] (b10)      [above=10mm of n01]     {$1$};
%    \node          (inv)      [below=5mm of  n01]     {\ };
%    \node [neuron] (n02)      [below=10mm of n01]     {$f_4$};

    \node          (i02)      [below=10mm of i01]      {$x_2$};
    

%    \node [neuron] (n03)      [right=20mm of n01]     {$f_3$};
%    \node          (inv)      [below=5mm of  n03]     {\ };
%    \node [neuron] (n04)      [right=20mm of n02]     {$f_4$};

%    \node [neuron] (n03)      [right=20mm of inv]     {$f_{14}$};
    
%    \node (out)      [right=20mm of n01] {$o$};
    
    \draw [->] (b00) -- (n01) node[sloped, midway, above] {$w_3$} ;
    \draw [->] (i01) -- (n01) node[midway, above] {$w_1$} ;
    \draw [->] (i02) -- (n01) node[sloped, midway, above] {$w_2$} ;
    
%    \draw [->] (b00) -- (n02);
%    \draw [->] (i01) -- (n02);
%    \draw [->] (i02) -- (n02);
%    
%    \draw [->] (n01) -- (n03);
%    \draw [->] (n02) -- (n03);
%    
%    \draw [->] (n01) -- (n04);
%    \draw [->] (n02) -- (n04);
%    
%    \draw [->] (n01) -- (n03);
%    \draw [->] (n02) -- (n03);
    
%    \draw [->] (n01) -- (out);
    
  \end{tikzpicture}

  \end{center}



\begin{align*}
f(x_1, x_2) &= \tanh(x_1 \times w_1 + x_2 \times w_2) \\
\end{align*}

}




\begin{frame}

\note{
This presentation was created and narrated by Lane Schwartz. \\ \ \\
You are free to reproduce and adapt this work under the terms of the Creative Commons Attribution-ShareAlike 4.0 International License.
}

\begin{center}
\includegraphics[scale=0.3]{images/CreativeCommons} \ \ \ \ \ 
\includegraphics[scale=0.3]{images/CreativeCommonsAttribution} \ \ \ \ \ 
\includegraphics[scale=0.3]{images/CreativeCommonsSharealike}
\end{center}

\ \\

%\setbeamertemplate{itemize items}[$+$]
\begin{itemize}
\item[\textbullet] This presentation was created and narrated by Lane Schwartz. \\ \ \\
\item[\textbullet] You are free to reproduce and adapt this work under the terms of the Creative Commons Attribution-ShareAlike 4.0 International License.
\end{itemize}

\end{frame}


%\begin{frame}
\frametitle{Linear unit}

\begin{equation*}
y = b + \sum_i x_i w_i
\end{equation*}


\begin{tikzpicture}
\begin{axis}[ylabel=output,xlabel=weighted input,scale=0.3,grid=major,ytick={0},xtick={0},xmin=-5,xmax=5,ymin=-5,ymax=5]
  \addplot[blue, ultra thick][domain=-10:10] (x,x);
\end{axis}
\end{tikzpicture}

\end{frame}


%\input{tmp}


\end{document}
